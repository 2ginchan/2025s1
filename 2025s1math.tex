\documentclass[a4paper,12pt,dvipdfmx,fleqn, oneside]{jsarticle}
\usepackage{amssymb,amsthm,amsmath,ascmac,color,bm,enumerate,float,graphicx,tikz,framed}

\usetikzlibrary{intersections, calc}

 %フッタ中央.ページ番号を表示

\newtheoremstyle{qesstyle}
    {0pt}
    {0pt}
    {}
    {0pt}
    {\bfseries}
    {}
    { }
    {}
\newtheoremstyle{thmx}
    {}
    {}
    {}
    {}
    {\bfseries}
    {}
    { }
    {\thmname{#1}\thmnumber{ #2}\thmnote{ #3}}
\newtheoremstyle{defstyle}
    {}
    {}
    {}
    {}
    {\bfseries}
    {}
    { }
    {\thmname{#1}\thmnumber{ #2}\thmnote{ (#3)}}
\theoremstyle{defstyle}
\newtheorem{dfn}{定義}[section]
\newtheorem{thm}[dfn]{命題}
\theoremstyle{thmx}
\newtheorem*{thmx}{命題}
\theoremstyle{qesstyle}
\newtheorem{qes}[dfn]{問題}
\newtheorem{cat}[dfn]{注意}
\newtheorem*{Proof}{証明.}

\renewcommand{\theenumi}{\arabic{enumi}}
\renewcommand{\labelenumi}{(\theenumi)}

\everymath{\displaystyle}

\definecolor{shadecolor}{gray}{0.80}

\title{数理科学基礎過去問解答解説}
\author{}
\date{}

\begin{document}
\setlength{\abovedisplayskip}{4pt}
\setlength{\belowdisplayskip}{4pt}
\maketitle

\section*{はじめに}
これは,数理科学基礎の過去問のうち,微積パートに該当する\textbf{問題}とその\textbf{解答解説}をまとめたものである。同じ林先生のものを選んであるから,本プリントの勉強は試験にある程度役立つことが期待される。なお,不備の指摘を含めた質問は遠慮なく作成者まで。
\section*{問題一覧}
\begin{shadebox}
    \textbf{問題1}

    次の方程式を満たす$x$を求めよ。
    \begin{align*}
        \arccos \frac{5}{7} = \arcsin x- \arccos \frac{1}{5}
    \end{align*}
\end{shadebox}
\begin{shadebox}
    \textbf{問題2}

    関数$f(x)=\left|x^3\right|$が$C^n$級となる最大の自然数$n$を求めよ。ただし,$y=f(x)$が$C^{\infty}$級になる場合には,$n=\infty$と答えよ。
\end{shadebox}
\begin{shadebox}
    \textbf{問題3}

    次の関数の原始関数を一つ見つけよ。
    \begin{align*}
        \arctan x
    \end{align*}
\end{shadebox}
\begin{shadebox}
    \textbf{問題4}

    関数$f:(-\pi/ 2, \pi / 2) \rightarrow (- \infty , -1) \cup (1, \infty)$を,$f(x)=1/\sin x$と定めるとき,$f$の逆関数の導関数を求めよ。
\end{shadebox}
\begin{shadebox}
    \textbf{問題5}

    逆双曲正弦関数 $\text{arsinh} : \mathbb{R} \rightarrow \mathbb{R}$を考える。
    \begin{enumerate}
        \item 等式$\text{arsinh}\, x = \ln(x+\sqrt{x^2+1})$を示せ。
        \item $\frac{d}{dx} \text{arsinh} \, x$を求めよ。
    \end{enumerate}
\end{shadebox}
\begin{shadebox}
    \textbf{問題6}

    微分可能な関数$y=f(x)$に関する微分方程式
    \begin{align*}
        y'=2 \sin \frac{y}{2}
    \end{align*}
    について考える。
    \begin{enumerate}
        \item $\cos \frac{y}{2}=u$とおいて置換積分の公式を用いることにより,解を任意定数を含む形で表せ。
        \item $f(0)= \pi $を持つ解を,逆三角関数と双曲線関数により表せ。
        \item (2)で求めた解$y=f(x)$の導関数を求め,$\cosh x$で表せ。
    \end{enumerate}
\end{shadebox}
\begin{shadebox}
    \textbf{問題7}

    つぎの二変数関数$f: \mathbb{R}^2 \rightarrow \mathbb{R}$
    \begin{align*}
        f(x,y) =
        \begin{cases}
            \frac{x^3-x^3}{x^2+y^2} & (x,y) \neq (0,0) \\
            0                       & (x,y)= (0,0)
        \end{cases}
    \end{align*}
    を考える。
    \begin{enumerate}
        \item 関数$f(x,y)$の$(0,0)$におけるすべての偏微分係数を求めよ。
        \item 関数$f(x,y)$のグラフの点$(1,1,f(1,1))$における接平面の方程式を求めよ。
        \item 単位ベクトル $\bf{e}=
                  \begin{pmatrix}
                      (\sqrt{2})^{-1} \\
                      (\sqrt{2})^{-1}
                  \end{pmatrix}$に対し,関数$f(x,y)$の$(0,0)$における$\bf{e}$方向の微分係数を求めよ。
        \item 関数$f(x,y)$は$(0,0)$において微分可能かどうか判定せよ。
    \end{enumerate}
\end{shadebox}
\begin{shadebox}
    \textbf{問題8}

    次の関数$f: \mathbb{R} \rightarrow \mathbb{R}$を考える。関数$f$のすべての連続点からなる集合を$S(f)$とする。
    \begin{align*}
        f(x) =
        \begin{cases}
            x-x^3 & x \in \mathbb{Q}                       \\
            0     & x \in \mathbb{R} \backslash \mathbb{Q}
        \end{cases}
    \end{align*}
    \begin{enumerate}
        \item $S(f)$を求めよ。
        \item 任意の$x \in S(f)$において,$f$は連続であることを示せ。
        \item 任意の$x \in \mathbb{R} \backslash S(f)$において,$f$は不連続であることを示せ。
    \end{enumerate}
\end{shadebox}
なお,問題1,2,3,4,5は基本問題として小問集合になっていたものである。

\section*{解答解説}
\begin{shadebox}
    \textbf{問題1}

    次の方程式を満たす$x$を求めよ。
    \begin{align*}
        \arccos \frac{5}{7} = \arcsin x- \arccos \frac{1}{5}
    \end{align*}
\end{shadebox}

\begin{screen}
    \textbf{解答}

    \text{ }$\phi = \arccos \frac{5}{7}, \psi = \arccos \frac{1}{5}, \theta = \arcsin x$とおく。$\phi , \psi \in [0, \pi]$に注意すると,
    \begin{align*}
        \sin \phi = \sqrt{1- \cos ^2 \phi }= \sqrt{1- \left(\frac{5}{7}\right)^2}=\frac{2\sqrt{6}}{7} \\
        \sin \psi = \sqrt{1- \cos ^2 \psi }= \sqrt{1-\left(\frac{1}{5}\right)^2}= \frac{2\sqrt{6}}{5}
    \end{align*}
    であるから,
    \begin{align*}
        x = \sin \theta = \sin (\phi + \psi ) & = \sin \phi \cos \psi + \cos \phi \sin \psi                                                 \\
                                              & =\frac{2\sqrt{6}}{7} \frac{1}{5} + \frac{5}{7} \frac{2\sqrt{6}}{5} = \frac{12\sqrt{6}}{35}.
    \end{align*}
\end{screen}

\textbf{解説}

頻出問題である。似たような問題が複数出題されている。逆三角関数で気を付けるべきは,その値域であった。今回は$\phi , \psi$の範囲が制限されることにより$\sin$の値が一意に定まることに注意する。あとは特段注意することはない。

\begin{shadebox}
    \textbf{問題2}

    関数$f(x)=\left|x^3\right|$が$C^n$級となる最大の自然数$n$を求めよ。ただし,$y=f(x)$が$C^{\infty}$級になる場合には,$n=\infty$と答えよ。
\end{shadebox}

\begin{screen}
    \textbf{解答}

    \text{ }関数$f(x)$は,
    \begin{align*}
        f(x)= |x^3| = \begin{cases}
            x^3  & x \geq 0 \\
            -x^3 & x < 0
        \end{cases}
    \end{align*}
    である。明らかに,$x \neq 0$では$f(x)$は任意の回数微分可能でその導関数は連続であるから,$x= 0$周りでの振る舞いを調べる。
    \begin{align*}
        \lim_{x \to \pm 0} \frac{f(x)-f(0)}{x-0} = \lim_{x \to \pm 0} \frac{\pm x^3}{x} = \lim_{x \to \pm 0} (\pm x^2) =0
    \end{align*}
    ゆえ$f'(0)$であり,あわせて
    \begin{align*}
        f'(x) = \begin{cases}
            3x^2   & x  \geq 0 \\
            - 3x^2 & x <0
        \end{cases}
    \end{align*}
    となる。さらに,
    \begin{align*}
        \lim_{x \to \pm 0} \frac{f'(x)-f'(0)}{x-0} = \lim_{x \to \pm 0} \frac{\pm 3x^2}{x} = \lim_{x \to \pm 0} (\pm 3x) =0
    \end{align*}
    ゆえ$f''(0)$であり,あわせて
    \begin{align*}
        f''(x) = \begin{cases}
            6x  & x \geq 0 \\
            -6x & x <0
        \end{cases}
    \end{align*}
    となり,$f(x)$は$C^2$級。しかし,
    \begin{align*}
        \lim_{x \to \pm 0} \frac{f''(x)-f''(0)}{x-0} = \lim_{x \to \pm 0} \frac{\pm 6x}{x} = \pm 6
    \end{align*}
    ゆえ$f''(x)$は$x=0$で微分不可能であり,$f(x)$は$C^3$級ではない。

    \text{ }よって求める$n$は$n=2$。
\end{screen}

\textbf{解説}

数理科学基礎の共通資料の確認問題そのままである。上記解答は丁寧に書いたが,$f''(x)=6|x|$が$x=0$で微分不可能であることは直接述べても問題ないだろう。共通資料の解答もそのように書いてある。

\begin{shadebox}
    \textbf{問題3}

    次の関数の原始関数を一つ見つけよ。
    \begin{align*}
        \arctan x
    \end{align*}
\end{shadebox}

\begin{screen}
    \textbf{解答}

    \text{ }部分積分を行って,
    \begin{align*}
        \int \arctan x \, dx & = x \arctan x - \int \frac{x}{1+x^2} \, dx        \\
                             & = x \arctan x - \frac{1}{2} \ln (1+ x^2) + const.
    \end{align*}
    ゆえ,例えば原始関数として $x \arctan x - \frac{1}{2} \ln (1+ x^2).  $
\end{screen}

\textbf{解説}

$\arctan x$のままでは積分しにくいから,部分積分で微分して簡単にすることを考える。有利関数に落ちつくはずであるから,当然積分実行可能である。

\begin{shadebox}
    \textbf{問題4}

    関数$f:(-\pi/ 2, \pi / 2) \rightarrow (- \infty , -1) \cup (1, \infty)$を,$f(x)=1/\sin x$と定めるとき,$f$の逆関数の導関数を求めよ。
\end{shadebox}

\begin{screen}
    \textbf{解答}

    \text{ }$y=f(x)$即ち$x=f^{-1}(y)$とすると,$y=1/\sin x $と $x \in (-\pi /2 , \pi /2)$から$\cos x = \sqrt{1-1/y^2}$であることに注意する。また,
    \begin{align*}
        f'(x)=- \frac{\cos x}{\sin^2 x}
    \end{align*}
    と逆関数定理より
    \begin{align*}
        (f^{-1}(y))' & = \frac{1}{f'(x)}
        = - \frac{\sin ^2 x}{\cos x}              = - \frac{1/y^2}{\sqrt{1-1/y^2}} \\
                     & = - \frac{1}{|y|\sqrt{y^2-1}}.
    \end{align*}
    よって
    \begin{align*}
        (f^{-1}(x))' = - \frac{1}{|x|\sqrt{x^2-1}}.
    \end{align*}
\end{screen}
\newpage
\textbf{解説}

授業で$\sec x = 1/ \cos x$の逆関数の導関数を求めたから,それと何も変わりない。定義域の制限より,$\cos x$が正であることに注意すること。

\begin{shadebox}
    \textbf{問題5}

    逆双曲正弦関数 $\text{arsinh} : \mathbb{R} \rightarrow \mathbb{R}$を考える。
    \begin{enumerate}
        \item 等式$\text{arsinh}\, x = \ln(x+\sqrt{x^2+1})$を示せ。
        \item $\frac{d}{dx} \text{arsinh} \, x$を求めよ。
    \end{enumerate}
\end{shadebox}

\begin{screen}
    \textbf{解答}

    \begin{enumerate}
        \item 実際に計算してみれば,
              \begin{align*}
                  \sinh \ln (x + \sqrt{x^2 + 1}) & = \frac{1}{2}\left(x + \sqrt{x^2 +1 } - \frac{1}{x+ \sqrt{x^2 + 1}}\right) \\
                                                 & = \frac{1}{2}(x + \sqrt{x^2 + 1} - (\sqrt{x^2 + 1}-x))                     \\
                                                 & = x
              \end{align*}
              であるから,表式が従う。
        \item (1)を用いれば,
              \begin{align*}
                  \frac{d}{dx} \text{arsinh }\, x & = \frac{d}{dx} \ln (x + \sqrt{x^2+1})                         \\
                                                  & = \frac{1+\frac{x}{\sqrt{x^2+1}}}{x+\sqrt{x^2+1}}             \\
                                                  & = \frac{\sqrt{x^2 + 1} + x}{(x + \sqrt{x^2+1})\sqrt{x^2 + 1}} \\
                                                  & = \frac{1}{\sqrt{x^2+1}}.
              \end{align*}
    \end{enumerate}
\end{screen}

\textbf{解説}

数理科学基礎の共通資料の確認問題そのままである。なお,(1)の表式の右辺を得るには,次のようにする。$x = \sinh y$を$y$について解く。
\begin{align*}
    x = \sinh y \Leftrightarrow x = \frac{e^y - e^{-y}}{2} & \Leftrightarrow (e^y)^2 - 2 x e^y -1 = 0 \, (e^y > 0\text{を掛けて整理した}) \\& \Leftrightarrow e^y = x + \sqrt{x^2 + 1} \, (\because e^y >0) \\ & \Leftrightarrow y = \ln (x+ \sqrt{x^2 + 1})
\end{align*}
また,(2)は逆関数定理を用いても計算できる。$ y = \text{arsinh}, x $即ち$x = \sinh y$のもとで,
\begin{align*}
    \frac{dy}{dx} = \frac{1}{\frac{dx}{dy}}= \frac{1}{\cosh y} = \frac{1}{\sqrt{1 + \sinh y }}= \frac{1}{\sqrt{1+x^2}}
\end{align*}
である。

\begin{shadebox}
    \textbf{問題6}

    微分可能な関数$y=f(x)$に関する微分方程式
    \begin{align*}
        y'=2 \sin \frac{y}{2}
    \end{align*}
    について考える。
    \begin{enumerate}
        \item $\cos \frac{y}{2}=u$とおいて置換積分の公式を用いることにより,解を任意定数を含む形で表せ。
        \item $f(0)= \pi $を持つ解を,逆三角関数と双曲線関数により表せ。
        \item (2)で求めた解$y=f(x)$の導関数を求め,$\cosh x$で表せ。
    \end{enumerate}
\end{shadebox}
\begin{screen}
    \textbf{解答}
    \begin{enumerate}
        \item $y = 2n \pi \, (n \in \mathbb{Z})$ は自明な解であるから,これは除いて考える。このとき,
              \begin{align*}
                  \frac{1}{2 \sin \frac{y}{2}} y' = 1
              \end{align*}
              の両辺に$\int \, dx$をかぶせて,
              \begin{align*}
                   & \int \frac{1}{2 \sin \frac{y}{2}} y' \, dx = \int  \, dx         \\
                   & \int \frac{dy}{2 \sin \frac{y}{2}} = x - C \, (C \in \mathbb{R})
              \end{align*}
              を得る。左辺の積分を実行するため$\cos \frac{y}{2} = u$とおくと,$- \frac{1}{2}\sin \frac{y}{2} dy = du$で,$\sin^2 \frac{y}{2} = 1- u^2$に注意して
              \begin{align*}
                  \int \frac{dy}{2 \sin \frac{y}{2}} = \int -\frac{du}{1-u^2} = - \text{artanh}\,  u+ const.
              \end{align*}
              となる。従って元の微分方程式の解は任意定数$C$を用いて,
              \begin{align*}
                         & -\text{artanh} \, (\cos \frac{y}{2})  = x - C             \\
                  i.e.\, & y                             = 2 \arccos (\tanh (-x+C)).
              \end{align*}
    \end{enumerate}
\end{screen}
\begin{screen}
    \begin{enumerate}
        \item[(2)] (1)の一般解で$f(0)=\pi$とすれば,
            \begin{align*}
                2\arccos(\tanh C)     & = \pi \\
                \therefore \, \tanh C & =0    \\
                \therefore \, C       & = 0
            \end{align*}
            であるから,求める解は
            \begin{align*}
                f(x)=2\arccos(\tanh (-x)).
            \end{align*}
        \item[(3)] (2)の$f(x)$を微分すれば,
            \begin{align*}
                \frac{d}{dx} f(x) = \frac{-2(\tanh (-x))'}{\sqrt{1-\tanh^2 (-x)}} = \frac{2}{\cosh^2 x} |\cosh x| = \frac{2}{\cosh x}.
            \end{align*}
            ただし,$(\tanh(-x))'=-\frac{1}{\cosh^2 (-x)}=-\frac{1}{\cosh^2 x}$,$1-\tanh^2(-x)=\\1-\tanh^2 x = \frac{1}{\cosh^2 x} $,$\cosh x >0$を用いた。
    \end{enumerate}
\end{screen}

\textbf{解説}

与えられた微分方程式は変数分離型であるからただの積分の問題になる。その「ただの積分」も問題文で置換の方法が支持されており優しい。ただし,$\frac{1}{1-x^2}$の積分は部分分数分解をするのではなく逆双曲線関数を用いる方がよい\footnote{$u$について積分を実行したのち,$u$について解かなければならないためである。部分分数分解をして積分するとこれが煩雑になる。}。(2)の問題文もそれを示唆している。

(2)は優しいが,やはり逆三角関数の取り扱いにはいくら注意してもし過ぎるということはない。途中,$\arccos(\tanh C)=\pi /2$の両辺の$\cos$をとって$\tanh C=0$を演繹しているがこれが問題ないことは大丈夫であろうか。ポイントは,$\pi /2$が$\arccos$の値域に入っていることと,逆向きの$\arccos (\cos \theta) = \theta$が常には成り立たないことである\footnote{$\theta = 2\pi$が反例}。$y=\cos (\arccos x)$と$y = \arccos (\cos x)$のグラフをそれぞれ考えてみるのがよいだろう。

(3)も微分を行うだけで,双曲線関数の扱いに慣れれば優しいだろう。あるいは,$f(x)$の満たす微分方程式から,
\begin{align*}
    f'(x) = 2\sin (f(x)/2)= 2 \sin (\arccos (\tanh (-x))) = 2 \sqrt{1- \tanh^2(-x)}
\end{align*}
として,その後も同様の変形により同じ関数を得る。ただし,$\sin (\arccos x) = \sqrt{1-x^2}$であることには,符号を含めて注意が必要である。
\end{document}